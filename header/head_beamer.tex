% =====================================================================
% === Beamer Header ===================================================
% =====================================================================
% --- Filename:     beamerheader.tex
% --- Author:       Fabian Grünig
% ---               fabian@mathphys.fsk.uni-heidelberg.de
% --- Created:      2011-12-19
% --- Description:  Header für Beamer-Präsentationen
% ---               auch org-mode
% ---

% ---------------------------------------------------------------------
% --- Header ----------------------------------------------------------
% ---------------------------------------------------------------------

\documentclass[
  t, 
  ngermanb
]{beamer}

\usepackage[ngermanb]{babel}
% \usepackage{unicode-math, fontspec}

\usepackage{amsmath}
\usepackage{amssymb}
\usepackage{amsfonts}
\usepackage{amsthm}
\usepackage{mathtools}
\usepackage{graphicx}
\usepackage{ulem}

\usepackage{tikz}
  \usetikzlibrary{matrix}
  \usetikzlibrary{fit}
  \usetikzlibrary{backgrounds}
  \usetikzlibrary{arrows}
  \usetikzlibrary{shapes}
  \usetikzlibrary{positioning}
  \usetikzlibrary{mindmap}
  \usetikzlibrary{decorations.markings}
  \usetikzlibrary{decorations.pathreplacing}

\usepackage{calc}
\usepackage{ifthen}
\usepackage{xcolor}

\usepackage{bookmark}
\usepackage{pifont}
\usepackage{pgfpages}
\usepackage{listings}

\usepackage[T1]{fontenc}
\usepackage[utf8]{inputenc}
\usepackage{microtype}

% ---------------------------------------------------------------------
% --- Design ----------------------------------------------------------
% ---------------------------------------------------------------------

  \mode<presentation>{
%      \useoutertheme[subsection=false]{smoothbars}
      \useinnertheme{circles} % rectangles, circles, rounded
      \usecolortheme[RGB={153,0,0}]{structure}
      \definecolor{unihd}{RGB}{153,0,0}
      \definecolor{dark}{RGB}{115,0,0}
      \definecolor{light}{RGB}{241,229,229}
      \usecolortheme{whale}
		 \usecolortheme{orchid}
%	   \usecolortheme{beaver}
      \usefonttheme{professionalfonts}
      \setbeamercovered{transparent}
      \beamertemplatenavigationsymbolsempty
%      \setbeameroption{show notes on second screen}
      \setbeamertemplate{note page}[plain]
      \setbeamertemplate{footline}[frame number]
}

% \mode<presentation>{%
%   \definecolor{unihd}{RGB}{153,0,0}

%   \usecolortheme[RGB={153,0,0}]{structure}
%   \useoutertheme[subsection=false,compress]{smoothbars}
%   \useinnertheme{circles}
%   \usecolortheme{whale}

%   \setbeamertemplate{items}[circle]
%   \setbeamercovered{transparent}
%   \setbeamertemplate{navigation symbols}{}
% % \beamertemplatenavigationsymbolsempty
%   \setbeamertemplate{note page}[plain]
% }

% ---------------------------------------------------------------------
% --- Daten -----------------------------------------------------------
% ---------------------------------------------------------------------

%
% Dies ist eine Leiche aus Zeiten, in denen ich org-mode für 
% Präsentationen verwenden wollte. Das war eine doofe Idee.
% Ich war zu faul, dieses Setup zu ändern :(
%

% --- Definiere folgende Variablen in deinem tex-file
% 
%     \newcommand{\mytitle}{  }
%     \newcommand{\myshorttitle}{  }
%     \newcommand{\myauthor}{  }
%     \newcommand{\myshortauthor}{  }
%     \newcommand{\myinstitute}{  }
%     \newcommand{\myshortinstitute}{  }
%     \newcommand{\mysubject}{  }
%     \newcommand{\mykeywords}{  }
%     \newcommand{\mypagelayout}{  }
%     \newcommand{\mystartview}{  }
%     \newcommand{\mydate}{ }
%
% ---


