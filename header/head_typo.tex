% =====================================================================
% === LaTeX Header: Koma Article  =====================================
% =====================================================================
% ---
% ---
% ---

% === Typografie & Typeface  ==========================================

% --- Microtype!

\usepackage[
  expansion=true,
  protrusion=true
]
{microtype}

% --- Lines, Paragraphs

\usepackage[onehalfspacing]{setspace}

\setlength{\parindent}{1em}%{3ex}
\setlength{\parskip}{0ex}%{2ex}
\setlength{\parfillskip}{1em plus .75\textwidth}%{3ex}

% --- Typeface

\usepackage[oldstylenums,largesmallcaps]{kpfonts}

% - Koma Faces

\setkomafont{sectioning}{\normalfont\bfseries}
\setkomafont{pagehead}{\normalfont\normalcolor\footnotesize}
\setkomafont{pagefoot}{\normalfont\normalcolor\footnotesize}

% \setkomafont{chapter}{\normalfont\Huge\bfseries}
% \setkomafont{section}{\normalfont\huge\bfseries}
% \setkomafont{subsection}{\normalfont\Large\bfseries}
% \setkomafont{subsubsection}{\normalfont\large\bfseries}

% --- Tables

\usepackage{booktabs}

% --- Notes

\usepackage[
   bottom,      % Footnotes appear always on bottom. This is necessary
                % especially when floats are used
   stable,      % Make footnotes stable in section titles
   perpage,     % Reset on each page
   %para,       % Place footnotes side by side of in one paragraph.
   %side,       % Place footnotes in the margin
   ragged,      % Use RaggedRight
   %norule,     % suppress rule above footnotes
   multiple,    % rearrange multiple footnotes intelligent in the text.
   %symbol,     % use symbols instead of numbers
]{footmisc}

\deffootnote{1.5em}{1em}{\makebox[1.5em][l]{\thefootnotemark}}
\addtolength{\skip\footins}{\baselineskip}
\setlength{\dimen\footins}{10\baselineskip}
\interfootnotelinepenalty=10000

% --- Listings (Source Code)

% \usepackage{listings}
% \lstset{
%         basicstyle=\small\ttfamily, % Standardschrift
%         numbers=left,               % Ort der Zeilennummern
%         numberstyle=\tiny,          % Stil der Zeilennummern
%         stepnumber=2,               % Abstand zwischen den Zeilennummern
%         numbersep=5pt,              % Abstand der Nummern zum Text
%         tabsize=2,                  % Groesse von Tabs
%         extendedchars=true,         %
%         breaklines=true,            % Zeilen werden Umgebrochen
% %        keywordstyle=[1]\textbf,    % Stil der Keywords
% %        keywordstyle=[2]\textbf,    %
% %        keywordstyle=[3]\textbf,    %
% %        keywordstyle=[4]\textbf,    %
%         stringstyle=\color{green!40!black!100}, % Farbe der String
%         showspaces=false,           % Leerzeichen anzeigen ?
%         showtabs=false,             % Tabs anzeigen ?
%         showstringspaces=false      % Leerzeichen in Strings anzeigen ?
% }
% \lstloadlanguages{
%         %bash
%         %Lua
%         %Python
%         %Octave
%         %Lisp
%         %Ruby
%         %TeX
%         %C
%         %C++
%         %XML
%         %HTML
% }